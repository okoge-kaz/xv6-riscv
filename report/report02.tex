\RequirePackage{plautopatch}
\RequirePackage[l2tabu, orthodox]{nag}

\documentclass[platex,dvipdfmx, titlepage]{jlreq} % for platex

\usepackage{graphicx}
\usepackage{bxtexlogo}
\usepackage{here}
\usepackage{url}

\usepackage{listings, jlisting, color}
\definecolor{OliveGreen}{rgb}{0.0,0.6,0.0}
\definecolor{Orenge}{rgb}{0.89,0.55,0}
\definecolor{SkyBlue}{rgb}{0.28, 0.28, 0.95}
\lstset{
  language={C}, % 言語の指定
  basicstyle={\ttfamily},
  identifierstyle={\small},
  commentstyle={\smallitshape},
  keywordstyle={\small\bfseries},
  ndkeywordstyle={\small},
  stringstyle={\small\ttfamily},
  frame={tb},
  breaklines=true,
  columns=[l]{fullflexible},
  numbers=left,
  xrightmargin=0zw,
  xleftmargin=0zw,
  numberstyle={\scriptsize},
  stepnumber=1,
  numbersep=1zw,
  lineskip=-0.5ex,
  keywordstyle={\color{SkyBlue}},     %キーワード(int, ifなど)の書体指定
  commentstyle={\color{OliveGreen}},  %注釈の書体
  stringstyle=\color{Orenge}          %文字列
}

\usepackage{tcolorbox}
\newtcbox{\code}[1][]{
  colback=gray!10!white,
  colframe=gray!20!white,
  boxrule=1pt,
  left=0mm,right=0mm,top=0mm,bottom=0mm,
  box align=base,
  nobeforeafter,
  fontupper=\ttfamily
}

\title{システムソフトウェア 第2回課題}

\author{学生番号 20B30790 藤井 一喜}
\date{\today}

\begin{document}

\maketitle

\section{ファイルシステムAPI強化}

\subsection{openへのO\_APPENDフラグの追加}

\subsubsection*{実装}

\code{kernel/fcntl.h}にO\_APPENDフラグを追加した。また\code{kernel/sysfile.c}に

\begin{lstlisting}[caption={kernel/fcntl.h}]
if((omode & O_APPEND) && ip->type == T_FILE){
  // seek to end of file
  f->off = ip->size;
}
\end{lstlisting}

\noindent を追加した。これにより、ファイルをオープンする際にO\_APPENDフラグが指定されている場合、オフセットがファイルの末尾に設定されるようになった。

\noindent 次に、リダイレクション \code{>>} を追加するために \code{user/sh.c} を以下のように変更した。

\begin{lstlisting}[caption={user/sh.c}]
    case '+':  // >>
        cmd = redircmd(cmd, q, eq, O_WRONLY|O_CREATE|O_APPEND, 1);
    break;
\end{lstlisting}

\subsubsection*{テスト}

以下のように、テストプログラムを作成し、\code{make qemu}後に\code{usertests}を実行することで、O\_APPENDフラグが正しく機能していることを確認した。
また、\code{>>} (リダイレクション)に関しては、\code{make qemu}後に、あらかじめ作成しておいたtest.txtに対して、実際に\code{echo hello >> test.txt}を実行し、ファイルの末尾に文字列が追加されていることを確認した。

\begin{lstlisting}[caption={user/usertests.c}]
// test O_APPEND.
void
append1(char *s)
{
  char buf[32];

  unlink("appendfile");
  int fd1 = open("appendfile", O_CREATE|O_WRONLY|O_TRUNC);
  write(fd1, "abcd", 4);
  close(fd1);

  int fd2 = open("appendfile", O_RDONLY);
  int n = read(fd2, buf, sizeof(buf));
  if(n != 4){
    printf("%s: read %d bytes, wanted 4\n", s, n);
    printf("file=%s\n", buf);
    exit(1);
  }
  close(fd2);

  fd1 = open("appendfile", O_WRONLY|O_APPEND);
  write(fd1, "efgh", 4);
  close(fd1);

  fd2 = open("appendfile", O_RDONLY);
  n = read(fd2, buf, sizeof(buf));
  if(n != 8){
    printf("%s: read %d bytes, wanted 8\n", s, n);
    printf("file=%s\n", buf);
    exit(1);
  }
  close(fd2);

  fd1 = open("appendfile", O_WRONLY|O_APPEND);
  write(fd1, "ijklmnop", 8);
  close(fd1);

  fd2 = open("appendfile", O_RDONLY);
  n = read(fd2, buf, sizeof(buf));
  if(n != 16){
    printf("%s: read %d bytes, wanted 16\n", s, n);
    printf("file=%s\n", buf);
    exit(1);
  }
  close(fd2);

  unlink("appendfile");
}

... (abbreviated) ...

  { "append1", append1 },
\end{lstlisting}

テストコードの内容としては、O\_APPENDフラグが正しく機能しているかどうかを確認するために、文字列をファイル末尾に追加する動作を繰り返し、ファイルの内容が正しいかどうかを確認するというものである。

usertestsを実行した結果、問題なくPASSした。

\subsection{システムコールlseekの追加}

\subsubsection*{実装}

system callを追加するために、\code{user/usys.pl}に \code{entry("lseek");}を追加し、\code{user/user.h}に \code{int lseek(int, int, int);}を追加、\code{kernel/sysycall.h}に以下を追加した。

\begin{lstlisting}[caption={kernel/sysycall.h}]
    #define SYS_lseek  23
\end{lstlisting}

\noindent また、\code{kernel/sysycall.c}に

\begin{lstlisting}[caption={kernel/sysycall.c}]
    extern uint64 sys_lseek(void);
    ...
    [SYS_lseek]   sys_lseek,
\end{lstlisting}
\noindent を追加、\code{kernel/fcntl.h}に以下を追加した。

\begin{lstlisting}[caption={kernel/fcntl.h}]
    #define SEEK_SET 0
    #define SEEK_CUR 1
    #define SEEK_END 2
\end{lstlisting}

\noindent また、\code{kernel/sysfile.c}に以下を追加した。

\begin{lstlisting}[caption={kernel/sysfile.c}]
  uint64
  sys_lseek(void)
  {
    int fd, offset, whence;
  
    struct file *f; // file pointer
    struct proc *p = myproc(); // current process
  
    argint(0, &fd);
    argint(1, &offset);
    argint(2, &whence);
  
    // fd < 0 : invalid file descriptor
    // fd >= NOFILE : invalid file descriptor
    // p->ofile[fd] == 0 : file is not opened
    if(fd < 0 || fd >= NOFILE || (f = p->ofile[fd]) == 0)
      return -1;
  
    // 操作対象が通常のファイルの場合
    if(f->type == FD_INODE){
      ilock(f->ip);
  
      // 更新前のoffsetを保存
      uint old_offset = f->off;
  
      if(whence == SEEK_SET) {
        f->off = offset; // offset from the beginning of the file (offsetの位置)
      } else if(whence == SEEK_CUR) {
        f->off += offset; // offset from the current position (現在のoffsetにoffsetを加えた値)
      } else if(whence == SEEK_END) {
        f->off = f->ip->size + offset; // offset from the end of the file (ファイルサイズにoffsetを加えた値)
      } else {
        iunlock(f->ip);
        return -1;
      }
  
      // 計算された新しいオフセットがファイルサイズより大きい場合
      // new_offset > file_size
      if(f->off > f->ip->size) {
        // file sizeを更新
        f->ip->size = f->off;
      }
  
      // new_offset < 0 OR new_offset > file_size limit
      if(f->off < 0 || f->off > MAXFILE * BSIZE) {
        // offsetを更新前の値に戻す
        f->off = old_offset;
        iunlock(f->ip);
        return -1;
      }
  
      iunlock(f->ip);
      return f->off;
    }
    return -1;
  }
\end{lstlisting}

方針としては、引数としてfd, offset, whenceを受け取った後、typeがFD\_INODEの場合に、\code{f->off}を変更することで、lseekを実現することにした。

オフセットを変更する際は、ilockによりinodeをロックする処理を加えた。

また、問題文にあるように whence には、\code{SEEK\_SET}、\code{SEEK\_CUR}、\code{SEEK\_END}のいずれかが入ることになるので、それぞれの場合に対応するように場合分けを施し、新しいオフセットを計算するようにした。

また求められている仕様に従い、新しいオフセットがファイルサイズより大きい場合は、ファイルサイズを更新するようにした。

実装全体は、私の個人リポジトリにある。(\url{https://github.com/okoge-kaz/xv6-riscv})

\subsubsection*{テスト}

以下のようなテストコードを追加した。

\begin{lstlisting}[caption={user/usertests.c}]
  // test lseek.
  void
  lseek1(char *s){
  
    unlink("seekfile");
    int fd1 = open("seekfile", O_CREATE|O_RDWR|O_APPEND);
    if (fd1 < 0) {
      printf("open failed\n");
      exit(1);
    }
    write(fd1, "hello", 6);
  
    int off1 = lseek(fd1, 0, SEEK_SET);// オフセットをファイルの先頭に戻す
    // off1 = 0
    if (off1 != 0) {
      printf("lseek failed\n");
      printf("off1=%d, expected 0\n", off1);
      exit(1);
    }
  
    char buf[6];
    int n = read(fd1, buf, 6);
    if (n != 6) {
      printf("read failed\n");
      printf("buf=%s, n=%d, expected=%d\n", buf, n, 6);
      exit(1);
    }
  
    int off2 = lseek(fd1, 200, SEEK_SET);// オフセットをファイルの先頭から200バイト目にセットする
    // off2 = 200
    if (off2 != 200) {
      printf("lseek failed\n");
      printf("off2=%d, expected 200\n", off2);
      exit(1);
    }
  
    int off3 = lseek(fd1, -200, SEEK_CUR);// オフセットを現在の位置から200バイト分戻す
    // off3 = 0
    if (off3 != 0) {
      printf("lseek failed\n");
      printf("off3=%d, expected 0\n", off3);
      exit(1);
    }
  
    int off4 = lseek(fd1, 0, SEEK_CUR);// オフセットを変更しない。現在のオフセットを得る
    // off3 == off4
    if (off3 != off4) {
      printf("lseek failed\n");
      printf("off3=%d, off4=%d, expected off3 == off4\n", off3, off4);
      exit(1);
    }
  
    int off5 = lseek(fd1, 0, SEEK_END);// オフセットをファイルの末尾まで進める。返値はファイルの大きさ
    // off5 = 200
    if (off5 != 200) {
      printf("lseek failed\n");
      printf("off5=%d, expected 200\n", off5);
      exit(1);
    }
  
    int off6 = lseek(fd1, 100, SEEK_END);// ファイル末尾に100バイト追加する
    // off6 = 300
    if (off6 != 300) {
      printf("lseek failed\n");
      printf("off6=%d, expected 300\n", off6);
      exit(1);
    }
  
    int off7 = lseek(fd1, -100, SEEK_CUR);// オフセットを100バイト戻す
    // off7 = 200
    if (off7 != 200) {
      printf("lseek failed\n");
      printf("off7=%d, expected 200\n", off7);
      exit(1);
    }
  
    unlink("seekfile");
  }
  
  void
  lseek2(char * s){
    unlink("seekfile");
    int fd1 = open("seekfile", O_CREATE|O_WRONLY);
    if (fd1 < 0) {
      printf("open failed\n");
      exit(1);
    }
    write(fd1, "Good bye!\n", 11);
    close(fd1);
  
    fd1 = open("seekfile", O_WRONLY);
    if (fd1 < 0) {
      printf("open failed\n");
      exit(1);
    }
  
    lseek(fd1, 5, SEEK_SET);
    write(fd1, "night!\n", 7);
    close(fd1);
  
    fd1 = open("seekfile", O_RDONLY);
    if (fd1 < 0) {
      printf("open failed\n");
      exit(1);
    }
  
    char buf[20];
    int n = read(fd1, buf, 20);
    if (n != 12) {
      printf("read failed\n");
      printf("buf=%s, n=%d, expected=%d\n", buf, n, 12);
      exit(1);
    }
  
    if (strcmp(buf, "Good night!\n") != 0) {
      printf("read failed\n");
      printf("buf=%s, expected=%s\n", buf, "Good night!\n");
      exit(1);
    }
  
    unlink("seekfile");
  }
  
  void
  lseek3(char * s){
    unlink("seekfile");
    int fd1 = open("seekfile", O_CREATE|O_WRONLY);
    if (fd1 < 0) {
      printf("open failed\n");
      exit(1);
    }
    write(fd1, "Hello World", 12);
    close(fd1);
  
    fd1 = open("seekfile", O_RDWR);
    int offset = lseek(fd1, 200, SEEK_SET);// オフセットをファイルの先頭から200バイト目にセットする
    if (offset != 200) {
      printf("lseek failed\n");
      printf("offset=%d, expected 200\n", offset);
      exit(1);
    }
  
    int file_size = lseek(fd1, 0, SEEK_END);// オフセットをファイルの末尾まで進める。返値はファイルの大きさ
    if (file_size != 200) {
      printf("lseek failed\n");
      printf("file_size=%d, expected 200\n", file_size);
      exit(1);
    }
  
    int error = lseek(fd1, -250, SEEK_CUR);// オフセットを現在の位置から250バイト分戻す (オフセットが負になる)
    if (error != -1) {// error
      printf("lseek failed\n");
      printf("error=%d, expected -1\n", error);
    }
  
    // errorが起きた場合は、オフセットは変化しない
    int offset2 = lseek(fd1, 0, SEEK_CUR);
    if (offset2 != 200) {
      printf("lseek failed\n");
      printf("offset2=%d, expected 200\n", offset2);
      exit(1);
    }
  
    close(fd1);
    unlink("seekfile");
  }
\end{lstlisting}

lseek1 のテストについては、ファイルディスクリプタのオフセットを移動させ、想定した通りの挙動であるかどうかを確認している。
この際、SEEK\_SET, SEEK\_CUR, SEEK\_END の挙動すべてを確認している

また、lseek2では、ファイル書き込みの際に、オフセットを移動させて書き込みを行うことで、想定する文字列が得られるかどうかを確認することを通じて、lseekの挙動を確認している。

lseek3では、仕様として求められている要件が充足されているかを確かめるために、新しいオフセットが負になるようなlseekの呼び出しや、ファイルサイズよりも大きなオフセットを指定するようなlseekの呼び出しを行い、検証を行っている。

この実装についても、個人リポジトリにて全体を公開している。(\url{https://github.com/okoge-kaz/xv6-riscv})

\subsection{得られた知見}

ファイルディスクリプタのオフセット値を変更することで、ファイルの読み書きを行う位置を変更できることが実感を伴って理解できた。

また、ファイルホールという概念についても理解を深めることができた。

\subsection{環境}

講義にて用意されている Docker Image を使用している。(docker-compose を用いて \code{docker compose up}にて起動できるようにして使用した。)

\begin{itemize}
    \item PC: MacStudio 2021 M1 Max
    \item Memory: 64 GB
    \item OS: macOS 13.0.1
\end{itemize}

\subsection{参考文献}

UNIX/Linux プログラミング教室 冨永和人, 権堂克彦

Linuxプログラミングインターフェース Michael Kerrisk

\end{document}
